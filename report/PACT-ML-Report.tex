% Options for packages loaded elsewhere
\PassOptionsToPackage{unicode}{hyperref}
\PassOptionsToPackage{hyphens}{url}
\PassOptionsToPackage{dvipsnames,svgnames,x11names}{xcolor}
%
\documentclass[
  12pt]{article}

\usepackage{amsmath,amssymb}
\usepackage{iftex}
\ifPDFTeX
  \usepackage[T1]{fontenc}
  \usepackage[utf8]{inputenc}
  \usepackage{textcomp} % provide euro and other symbols
\else % if luatex or xetex
  \usepackage{unicode-math}
  \defaultfontfeatures{Scale=MatchLowercase}
  \defaultfontfeatures[\rmfamily]{Ligatures=TeX,Scale=1}
\fi
\usepackage{lmodern}
\ifPDFTeX\else  
    % xetex/luatex font selection
\fi
% Use upquote if available, for straight quotes in verbatim environments
\IfFileExists{upquote.sty}{\usepackage{upquote}}{}
\IfFileExists{microtype.sty}{% use microtype if available
  \usepackage[]{microtype}
  \UseMicrotypeSet[protrusion]{basicmath} % disable protrusion for tt fonts
}{}
\makeatletter
\@ifundefined{KOMAClassName}{% if non-KOMA class
  \IfFileExists{parskip.sty}{%
    \usepackage{parskip}
  }{% else
    \setlength{\parindent}{0pt}
    \setlength{\parskip}{6pt plus 2pt minus 1pt}}
}{% if KOMA class
  \KOMAoptions{parskip=half}}
\makeatother
\usepackage{xcolor}
\setlength{\emergencystretch}{3em} % prevent overfull lines
\setcounter{secnumdepth}{5}
% Make \paragraph and \subparagraph free-standing
\ifx\paragraph\undefined\else
  \let\oldparagraph\paragraph
  \renewcommand{\paragraph}[1]{\oldparagraph{#1}\mbox{}}
\fi
\ifx\subparagraph\undefined\else
  \let\oldsubparagraph\subparagraph
  \renewcommand{\subparagraph}[1]{\oldsubparagraph{#1}\mbox{}}
\fi


\providecommand{\tightlist}{%
  \setlength{\itemsep}{0pt}\setlength{\parskip}{0pt}}\usepackage{longtable,booktabs,array}
\usepackage{calc} % for calculating minipage widths
% Correct order of tables after \paragraph or \subparagraph
\usepackage{etoolbox}
\makeatletter
\patchcmd\longtable{\par}{\if@noskipsec\mbox{}\fi\par}{}{}
\makeatother
% Allow footnotes in longtable head/foot
\IfFileExists{footnotehyper.sty}{\usepackage{footnotehyper}}{\usepackage{footnote}}
\makesavenoteenv{longtable}
\usepackage{graphicx}
\makeatletter
\def\maxwidth{\ifdim\Gin@nat@width>\linewidth\linewidth\else\Gin@nat@width\fi}
\def\maxheight{\ifdim\Gin@nat@height>\textheight\textheight\else\Gin@nat@height\fi}
\makeatother
% Scale images if necessary, so that they will not overflow the page
% margins by default, and it is still possible to overwrite the defaults
% using explicit options in \includegraphics[width, height, ...]{}
\setkeys{Gin}{width=\maxwidth,height=\maxheight,keepaspectratio}
% Set default figure placement to htbp
\makeatletter
\def\fps@figure{htbp}
\makeatother

\addtolength{\oddsidemargin}{-.5in}%
\addtolength{\evensidemargin}{-1in}%
\addtolength{\textwidth}{1in}%
\addtolength{\textheight}{1.7in}%
\addtolength{\topmargin}{-1in}%
\makeatletter
\@ifpackageloaded{caption}{}{\usepackage{caption}}
\AtBeginDocument{%
\ifdefined\contentsname
  \renewcommand*\contentsname{Table of contents}
\else
  \newcommand\contentsname{Table of contents}
\fi
\ifdefined\listfigurename
  \renewcommand*\listfigurename{List of Figures}
\else
  \newcommand\listfigurename{List of Figures}
\fi
\ifdefined\listtablename
  \renewcommand*\listtablename{List of Tables}
\else
  \newcommand\listtablename{List of Tables}
\fi
\ifdefined\figurename
  \renewcommand*\figurename{Figure}
\else
  \newcommand\figurename{Figure}
\fi
\ifdefined\tablename
  \renewcommand*\tablename{Table}
\else
  \newcommand\tablename{Table}
\fi
}
\@ifpackageloaded{float}{}{\usepackage{float}}
\floatstyle{ruled}
\@ifundefined{c@chapter}{\newfloat{codelisting}{h}{lop}}{\newfloat{codelisting}{h}{lop}[chapter]}
\floatname{codelisting}{Listing}
\newcommand*\listoflistings{\listof{codelisting}{List of Listings}}
\makeatother
\makeatletter
\makeatother
\makeatletter
\@ifpackageloaded{caption}{}{\usepackage{caption}}
\@ifpackageloaded{subcaption}{}{\usepackage{subcaption}}
\makeatother
\ifLuaTeX
  \usepackage{selnolig}  % disable illegal ligatures
\fi
\usepackage[]{natbib}
\bibliographystyle{agsm}
\usepackage{bookmark}

\IfFileExists{xurl.sty}{\usepackage{xurl}}{} % add URL line breaks if available
\urlstyle{same} % disable monospaced font for URLs
\hypersetup{
  pdftitle={PACT-ML: Coding United Nation Peacekeeping Data from reports to the Secretary-General},
  pdfauthor={Felix Kube},
  pdfkeywords={Machine Learning, Natural Language Processing, United
Nations Peacekeeping, BERT, roBERTa},
  colorlinks=true,
  linkcolor={blue},
  filecolor={Maroon},
  citecolor={Blue},
  urlcolor={Blue},
  pdfcreator={LaTeX via pandoc}}


\begin{document}


\def\spacingset#1{\renewcommand{\baselinestretch}%
{#1}\small\normalsize} \spacingset{1}


%%%%%%%%%%%%%%%%%%%%%%%%%%%%%%%%%%%%%%%%%%%%%%%%%%%%%%%%%%%%%%%%%%%%%%%%%%%%%%

\date{May 14, 2025}
\title{\bf PACT-ML: Coding United Nation Peacekeeping Data from reports
to the Secretary-General}
\author{
Felix Kube\thanks{Thanks for good advice during the semester, Chris and
Killian. Also all the peers I talked to while tackling this group
project alone :) it was not so alone after all.}\\
Data Science Lab, Hertie School\\
}
\maketitle

\bigskip
\bigskip
\begin{abstract}
The Peacekeeping Activity Dataset (PACT) is the first of its kind data
collection to shine light on what peacekeepers actually implement while
deployed. In the past, many projects have looked towards mandates to
study how specific tasks and mission success are related. PACT used
report data from the mission heads to the Secretary-General of the UN to
code up to 39 categories of task implementation on six different
engagement levels. This project, PACT-ML, aims to extend the data
collections of PACT 1.0 (\citet{Blair2022}) and PACT 2.0 (\citet{PACT2},
\citet{Otto2024}) by using selected reports of PACT 2.0 to examine the
application of Machine Learning / Natural Language Processing techniques
to automatically code this sort of data from the reports.
\end{abstract}

\noindent%
{\it Keywords:} Machine Learning, Natural Language Processing, United
Nations Peacekeeping, BERT, roBERTa
\vfill

\newpage
\spacingset{1.9} % DON'T change the spacing!

\section{Notes}\label{notes}

\begin{itemize}
\tightlist
\item
  ONUCA reports left out due to super old report format and issues in
  preprocessing.
\item
  maybe make cutoff at the introduction of the newer format?
\item
  leave out cross-country missions due to problems in data parsing and
  different language format (only coded on state-level, when that
  state's name is explicitely mentioned)

  \begin{itemize}
  \tightlist
  \item
    could also preprocess data and merge those states back together
  \end{itemize}
\end{itemize}

\section{Introduction}\label{sec-intro}

\begin{longtable}[]{@{}lllll@{}}
\caption{D-optimality values for design X under five different
scenarios.}\label{tbl-one}\tabularnewline
\toprule\noalign{}
one & two & three & four & five \\
\midrule\noalign{}
\endfirsthead
\toprule\noalign{}
one & two & three & four & five \\
\midrule\noalign{}
\endhead
\bottomrule\noalign{}
\endlastfoot
1.23 & 3.45 & 5.00 & 1.21 & 3.41 \\
1.23 & 3.45 & 5.00 & 1.21 & 3.42 \\
1.23 & 3.45 & 5.00 & 1.21 & 3.43 \\
\end{longtable}

\begin{itemize}
\tightlist
\item
  Note that figures and tables (such as Table~\ref{tbl-one}) should
  appear in the paper, not at the end or in separate files.
\end{itemize}

Motivation: Feasibility test to use data of this kind to automatically
code newly written reports.

Background on the research project itself

\section{Data}\label{sec-data}

PACT 2.0 data, parsed reports

The UNPKOs included are: MINUGUA, MINUJUSTH, MINUSTAH, MIPONUH, ONUCA,
ONUSAL, UNCPSG, UNCRO, UNMIBH, UNMIH, UNMIK, UNMISET, UNMIT, UNMOP,
UNMOT, UNOMIG, UNPREDEP, UNPROFOR, UNSMIH, UNTAC, UNTAES, UNTAET and
UNTMIH.

Include table of how many reports per mission in the whole dataset

Exclusion of very old reports due to issues in parsing

Exclusion of cross-country reports due to issues with the difference in
language and adjacent codings, which we ignore for our language model
approach.

Discuss how the data is structured (PACT 2.0)

Describe process how the data is parsed to extract the paragraphs

\subsection{Validity checks}\label{validity-checks}

To be able to use our models on the data, we need to identify which
paragraph text belongs to the specific coding in the PACT 2.0 data set.
We take the reported paragraph, from which the manual coders at the
University of Uppsala made their judgement, as ground truth. We do this
because it is highly unlikely that mistakes happened, as each paragraph
as per the UN reporting scheme carries its paragraph number at the
beginning, and coders had to mark the relevant sentences within the PDFs
before adding them to the database. On a side note, this fine-grained
data which would allow us to train our models beyond the scope here, is
only available for 143 out of 473 total reports. Three reports reported
no activity and can be therefore be left out for parsing.

To check if parsing was sucessful and our data quality is sufficient, we
compare the number of paragraphs extracted from each PDF with the number
of total paragraphs reported in the PACT 2.0 data, and only use the data
for our model if the number of paragraphs align.

\section{Methods}\label{sec-meth}

Don't take any of these section titles seriously. They're just for
illustration.

\section{Results}\label{sec-results}

\section{Conclusion}\label{sec-conc}

\phantomsection\label{supplementary-material}
\bigskip

\begin{center}

{\large\bf SUPPLEMENTARY MATERIAL}

\end{center}

\begin{description}
\item[Title:]
Brief description. (file type)
\end{description}


  \bibliography{bibliography.bib}


\end{document}
